\documentclass[a4paper, twoside, ngerman]{article}
\usepackage[utf8]{inputenc}
\usepackage{babel}
\usepackage{hyperref}


\title{Exposé der Bachelorarbeit}
\author{Lukas Hüller}
\date{Februar 2021}

\def

\begin{document}

\maketitle

Die Arbeit setzt sich mit der Fragestellung auseinander, welche Frontend-Interfaces zum Erstellen von Prozess-Automatisierungs-Robotern (RPAs) benötigt werden. Darüber hinaus wird in der Arbeit auf die technische Implementierung einer Datenstruktur eingegangen, die das nahtlose Wechseln zwischen zwei dieser Frontend-Interfaces ermöglicht.

Die Bachelorarbeit wird im Rahmen des Bachelorprojektes geschrieben. Ziel des Projektes ist die Entwicklung einer Open-Source RPA-Plattform  \footnote{\url{https://github.com/bptlab/ark_automate}}, die es zuvor definierten Personas ermöglicht, eigene, robotergestützte Automatisierungen zu erstellen. Basierend auf Interviews mit möglichen Endnutzern der Software, ergaben sich verschiedene Personas, die über unsere Plattform Roboter erstellen sollen. Da das fachliche Vorwissen der Nutzer jedoch unterschiedlich ist, sollen ihnen verschiedene Oberflächen zur Erstellung der Roboter zur Verfügung gestellt werden. Um weiterhin eine kollaborative Zusammenarbeit zwischen den Personas zu ermöglichen, bedarf es in der Software einer einheitlichen Repräsentation der Robotermodelle. Mit dieser \glqq Single Source of Truth\grqq{} soll die Datenintegrität zwischen den verschiedenen Robotermodellen bewahrt werden. \newline



\end{document}
