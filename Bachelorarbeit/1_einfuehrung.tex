\section{Einführung}
\label{sec:ein}

Unternehmen stehen unter immensem Druck, ihre Prozesse und Abläufe zu automatisieren. Oftmals ist die Robotic Process Automation, die in den vergangenen Jahren ein beeindruckendes Wachstum erfährt \cite{GartnerQuadrant},  hierfür die geeignete Automatisierungsstrategie. RPA steigert die Produktivität der Abläufe durch den Einsatz sogenannter Softwareroboter. Diese Roboter übernehmen die Aktivitäten des Nutzers in Desktop- oder Webanwendungen, indem sie seine Interaktion mit den Oberflächen eigenständig ausführen. Da in den automatisierten Prozessen lediglich die menschlichen Softwareanwender durch den Menschen imitierende Roboter ersetzt werden, bleibt der Prozess unverändert. Dadurch entfällt der hohe Aufwand der Prozessintegration. 

Durch das große Wachstum dieser Automatisierungstechnologie entstand eine Vielzahl von Plattformen, die das Erstellen der Softwareroboter ermöglichen. Jedoch haben die mit potenziellen Stakeholdern durchgeführten Interviews gezeigt, das Funktionen gewünscht werden, die keine auf dem Markt verfügbare Plattform abbildet.
Wie im Fazit auf der Seite \pageref{anforderungsanalyse:fazit} detailiert beschrieben, dominierte in den Interviews der Wunsch nach einer RPA-Plattform, mit der Roboter durch verschiedene, standardisierte Modellierungssprachen wie zum Beispiel der BPMN, der Flussdiagramm-Syntax oder als Ereignisgesteuerte Prozesskette (EPK) erstellt werden können. Dies soll das Konfigurieren der Roboter für Low-Code-Entwickler\footnote{Low-Code-Entwickler\\ Entwickler, die vor allem mit grafischen Programmieroberflächen wie z. B. Blockeditoren oder Modellierungssprachen anstatt von textbasierter Programmierung arbeiten.} ermöglichen. Alle in der Branche bekannten Plattformen unterstützten lediglich das Modellieren von Robotern in deren proprietären Oberflächen; standardisierte Notationen können bislang nicht zum Erstellen einer Automatisierung verwendet werden. 

In der ersten Forschungsfrage dieser Arbeit wird untersucht, ob sich Modellierungssprachen zur Erstellung von RPA-Robotern eignen. In der zweiten Forschungsfrage wird analysiert, wie das Erstellen desselben Roboters in verschiedenen Modellierungssprachen erfolgen kann, sodass jeder an der Automatisierung beteiligte Entwickler die ihm vertraute Modellierungssprache verwenden kann. Hierzu werden eine einheitliche Speicherlösung für RPA-Roboter sowie die zur Übersetzung in visuelle Repräsentationen notwendigen Parser vorgestellt.

Nach einer Einführung in die RPA, einem Überblick über verwandte Arbeiten sowie der Dokumentation der Befragungen, wird im Kapitel \ref{analyse_sprachen} untersucht, ob sich Modellierungssprachen  für die Entwicklung von RPA-Robotern eignen. Auf Grundlage dieser Betrachtungen ist im nachfolgenden Kapitel das Konzept der einheitlichen Speicherlösung beschrieben. Das Kapitel \ref{implementierung_ssot} erklärt, wie die beschriebenen Konzepte implementiert werden können. Abschließend werden die Implementierung evaluiert (Kap. \ref{evaluation_ssot}), die wesentlichen Erkenntnisse zusammengefasst sowie ein Überblick über offene Forschungsfragen (Kap. \ref{conclusion}) gegeben. 
