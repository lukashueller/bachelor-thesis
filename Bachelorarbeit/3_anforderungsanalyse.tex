\section{Anforderungen an eine RPA-Plattform} \label{anforderungsanalyse}

UiPath veröffentlichte 2018 eine Auflistung \cite{uiPathReq} von sieben Anforderungen, die eine moderne RPA-Plattform erfüllen sollte. Da die Veröffentlichung bereits einige Jahre zurückliegt und die Entwicklung der RPA-Plattformen stetig voranschreitet, wurden zu Beginn des Projektes verschiedene Stakeholder befragt. Ziel der elf Interviews war es, die Anforderungen an eine neue RPA-Plattform zu definieren und mit den Ergebnissen der UiPath-Studie abzugleichen. 

In der nachfolgenden Zusammenfassung werden die für diese Arbeit relevanten Informationen zur Interaktion der Stakeholder mit den Modellierungsoberflächen hervorgehoben.

\subsection{Dokumentation der Interviews}
\label{interviews}

In den Interviews wurden neben Lehrstuhlmitarbeitenden, die unter anderem an Themen rund um RPA forschen, Prozessexperten, vier RPA-Entwickler\footnote{RPA-Entwickler \\ haben auch beratende Funktionen, weshalb die Begriffe RPA-Consultant und RPA-Developer von nun an synonym verwendet werden.} sowie ein Manager befragt. Als Prozessexperten wurden aus der Verwaltung des Hasso-Plattner-Instituts eine Sekretärin zweier Fachgruppen sowie die Referentin für Forschung und Lehre interviewt. Als IT-Prozessexperte wurde ein Experte für Vulnerability Management des Cyber Defence Centers der Deutschen Telekom befragt, der seit über einem Jahr mit einem externen Team an softwaregestützten Prozessautomatisierungen arbeitet. Zudem wurde der General Manager und Chief Financial Officer (CFO) von Thermondo interviewt.

\clearpage
\subsubsection{Interview: Prozessexperten aus der Verwaltung}
Die Prozessexperten der Verwaltung betonten, dass es in ihrem Büroalltag einen großen Bedarf für die Automatisierung und Digitalisierung von Prozessen gibt. Die tägliche Arbeit ist zwischen einem und zwei Dritteln von monotonen Verwaltungsaufgaben dominiert, die den Angestellten weder Spaß bereitet, noch sie im gewünschten Umfang bei der Arbeit effektiv voranbringt. Es wurden zahlreiche Prozesse vorgestellt, die wiederkehrend und auch fehleranfällig sind. Leider sind viele dieser Prozesse nur teilweise digitalisiert, weshalb eine End-To-End-Automatisierung mittels RPA vorab eine Digitalisierung der Prozesse erfordert.

\subsubsection{Interview: Prozessexperten in IT-Projekten}
Der Cyber Defence Experte arbeitet seit sechs Monaten als Prozessexperte mit zwei externen Entwicklern zusammen, die für ihn alltägliche Routineaufgaben automatisieren. Er bemängelte, dass der Erstellungsprozess der Roboter sehr viel Zeit in Anspruch nimmt. Zu Beginn jedes Automatisierungsprojektes erläutert er den Entwicklern und Consultants welche Prozesse automatisiert werden sollen. Auf dieser Grundlage wird mit allen Beteiligten ein Programmablaufplan erstellt. Im Anschluss daran entwickeln die Consultants die Automatisierung mit ihrer Plattform und präsentieren den Roboter dem Kunden. Dieser hat abschließend jedoch keinen Zugriff auf die Automatisierung, was er bemängelt. Ebenso würde der Prozessexperte gern kleine Änderungen an den fertigen Automatisierungen in einem leicht zu bedienen Interface direkt vornehmen können. Auch dies ist mit den derzeit bestehenden Lösungen nicht möglich.

\subsubsection{Interview: RPA-Entwickler}
In umfangreichen Interviews wurden vier RPA-Developerbefragt, die  mehrjährige Erfahrung im Bereich der RPA besitzen. 

Die Entwickler arbeiten seit einigen Jahren mit der Software des Marktführers UiPath. Der Prozess zur Erstellung eines Softwareroboters beginnt mit der Prozessvorstellung durch den Kunden. Im Nachgang des Erstgespräches werden die Prozesse detailliert gezeichnet. Hierzu wird von den Befragten ausschließlich die Modellierungssprache BPMN verwendet.  Auf Grundlage der Prozessskizze wird mögliches Potenzial zur Optimierung der Abläufe analysiert, bevor es zur finalen Besprechung mit dem Kunden erfolgt. Danach beginnt die Implementierung des Roboters in UiPath. Hierbei merkten beide Entwickler an, dass das gedankliche Übersetzen eines in BPMN gezeichneten Prozesses in die proprietäre UiPath-Syntax unnötig kompliziert ist. Aus ihrer Sicht sollte es auch möglich sein, einen RPA-Roboter direkt in der BPMN-Syntax zu erstellen. Einer der  Entwickler sagte zudem, dass nach Fertigstellung des Roboters dieser noch dokumentiert werden muss. Dafür eignet sich seiner Meinung nach BPMN als Dokumentationssprache hervorragend. Da der final implementierte Roboter oftmals von der ursprünglichen Skizze abweicht, muss dieser zu Dokumentationszwecken erneut in BPMN skizziert werden. Gewünscht wurde auch eine Dokumentationshilfe, die die verwendeten RPA-Befehle direkt in die Dokumentation einfließen lässt. Solch eine Funktion sucht man bisher auf dem Markt vergeblich. 

Ebenso merkten zwei Entwickler an, dass die Oberfläche von UiPath sehr komplex und benutzerunfreundlich ist. Zwei Entwickler suchen daher seit langer Zeit nach einem Tool, dass sich leichter bedienen lässt. Leider umfassen diese Tools meist nicht alle benötigten Funktionen, weshalb schlussendlich doch wieder auf UiPath zurückgegriffen werden muss. 

Weiterhin wurde von einem Entwickler ein "Marktplatz"  für Roboter gewünscht, der es ermöglichen sollte, Roboter innerhalb der Organisation oder auch in einer Open-Source Community zu teilen. Zudem sollte es eine Administrationsoberfläche geben, von der aus zentral Aktualisierungen eingespielt werden können. 

\subsubsection{Interview: Manager \& CFO}
Seit mehreren Jahren ist der Manager bestrebt, analoge Prozesse im Heizungs- und Handwerkerwesen zu digitalisieren. Er hofft, dass seine Firma durch eine höhere Anzahl automatisierter Prozesse in Zukunft besser skalieren kann. Seine wichtigste Feststellung aus den vergangenen Jahren war, dass man in Deutschland meilenweit von vollständig „automatisierbaren“ Prozessen entfernt ist. Der Grund hierfür sei, dass sich softwarebasierte Prozesse aufgrund veralteter Software und somit fehlender Schnittstellen (APIs) nicht automatisieren lassen. Er zeigte sich begeistert von der für ihn unbekannten Zukunftstechnologie RPA und wünschte sich eine Plattform, mit der auch Prozessexperten Roboter erstellen können. Der Manager betonte, dass viele Bestrebungen nach verstärkter Automatisierung nicht nur an fehlender Digitalisierung scheitern. \glqq Entwickler und Prozessberater sind schlichtweg zu teuer\grqq{}, verriet der CFO.

\subsection{Fazit}
\label{anforderungsanalyse:fazit}
Die vielfältigen Antworten der Befragten waren grundlegend für die Definition der Anforderungen an die neue Plattform. Zusammenfassend lässt sich festhalten, dass in allen Branchen, die die Befragten repräsentieren, ein ausgesprochen großes Interesse an den Möglichkeiten der Automatisierung bestand, um die alltägliche Arbeit zu erleichtern und zu effektivieren. 
Die Mitarbeitenden der Verwaltung sagten uns, dass bei der Entwicklung der Plattform in jedem Fall ein besonderes Augenmerk auf die alltagstaugliche Bedienung und schnelle Erlernbarkeit gelegt werden sollte. Die RPA-Developer wünschten sich vor allem eine leicht zu bedienende RPA-Plattform, mit der bestenfalls direkt in BPMN Roboter skizziert, Automatisierungen erstellt und abschließend die Prozesse dokumentiert werden können. Thermondos CFO verdeutlichte, dass mit einer guten RPA-Plattform auch Prozessexperten Roboter erstellen können sollen. \glqq Die Kosten für RPA-Consultants sind für viele Firmen eine Herausforderung\grqq{}, so der Manager. 

Die Auswertung der Interviews zeigt, dass sich der Großteil der Anforderungen aus der UiPath-Studie in den aktuellen Wünschen der Entwickler wiederspiegelt. Die Studie erachtet es als zwingend notwendig, dass die Roboter durch die Mitarbeitenden der Fachabteilungen erstellt werden können. Ebenso erwähnt die Veröffentlichung die Anforderung eines Marktplatzes für Roboter, mit der die \glqq globale Anwender-Community\grqq{} von den fertiggestellten Automatisierungen profitiert. Die Auflistung von UiPath bestätigt die Notwendigkeit eines Administrations-Interfaces und beschreibt zudem noch die Management-Konsole, die es den Mitarbeitern erleichtern soll, die Automatisierungen zu starten, zu kontrollieren und zu validieren. Einzig die Nutzung von Künstlicher Intelligenz, die als eine Anforderung von UiPath beschrieben wurde, konnte nicht aus den Interviews abgeleitet werden.