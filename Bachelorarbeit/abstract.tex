\begin{abstract}
\textbf{Abstract.} 
Die Robotic Process Automation (RPA) hat in den vergangenen Jahren im Bereich der Automatisierungslösungen zunehmend an Bedeutung gewonnen. Der Markt stellt inzwischen eine Vielzahl von Plattformen zur Verfügung, mit denen die Automatisierungen erstellt werden können. Fast alle dieser Plattformen besitzen proprietäre Oberflächen zur Erstellung der Roboter, wodurch ein ständiges Übersetzen der modellierten und dokumentierten Prozesse aus bekannten Modellierungssprachen wie die Business Process Model and Notation (BPMN) oder Flowcharts in die Darstellungsform der Anwendungen notwendig ist. Dadurch erhöht sich der Aufwand für RPA-Entwickler und Prozessexperten in der Erstellung der Roboter. 

Die vorliegende Arbeit untersucht die für die Erstellung von RPA-Robotern geeigneten Notationen. Auf dieser Grundlage wird eine Architektur vorgestellt, die das Speichern der Roboter in einer einheitlichen Datenspeicherlösung ermöglicht. Zudem wird gezeigt, wie die verschiedenen Repräsentationen aus dieser \textit{\glqq Single Source of Truth\grqq{}} (SSoT) generiert werden können. Mit der vorgestellten Lösung ist es möglich, RPA-Roboter in verschiedenen Modellierungssprachen zu erstellen. Dadurch werden Prozessexperten sowie Low-Code-Developer in die Lage versetzt, direkt in ihrer gewohnten Modellierungsumgebung die Automatisierungen zu programmieren.

\end{abstract}